\documentclass[
11pt, % The default document font size, options: 10pt, 11pt, 12pt
codirector, % Uncomment to add a codirector to the title page
]{charter} 




% El títulos de la memoria, se usa en la carátula y se puede usar el cualquier lugar del documento con el comando \ttitle
\titulo{Diseño e implementación de un data logger para mediciones meteorológicas} 

% Nombre del posgrado, se usa en la carátula y se puede usar el cualquier lugar del documento con el comando \degreename
\posgrado{Carrera de Especialización en Sistemas Embebidos} 
%\posgrado{Carrera de Especialización en Internet de las Cosas} 
%\posgrado{Carrera de Especialización en Intelegencia Artificial}
%\posgrado{Maestría en Sistemas Embebidos} 
%\posgrado{Maestría en Internet de las cosas}

% Tu nombre, se puede usar el cualquier lugar del documento con el comando \authorname
\autor{Ing. Renato Barresi} 

% El nombre del director y co-director, se puede usar el cualquier lugar del documento con el comando \supname y \cosupname y \pertesupname y \pertecosupname
\director{Nombre del Director}
\pertenenciaDirector{pertenencia} 
% FIXME:NO IMPLEMENTADO EL CODIRECTOR ni su pertenencia
\codirector{John Doe} % para que aparezca en la portada se debe descomentar la opción codirector en el documentclass
\pertenenciaCoDirector{FIUBA}

% Nombre del cliente, quien va a aprobar los resultados del proyecto, se puede usar con el comando \clientename y \empclientename
\cliente{Javier Marin}
\empresaCliente{Tech Enterprise S.A}

% Nombre y pertenencia de los jurados, se pueden usar el cualquier lugar del documento con el comando \jurunoname, \jurdosname y \jurtresname y \perteunoname, \pertedosname y \pertetresname.
\juradoUno{Nombre y Apellido (1)}
\pertenenciaJurUno{pertenencia (1)} 
\juradoDos{Nombre y Apellido (2)}
\pertenenciaJurDos{pertenencia (2)}
\juradoTres{Nombre y Apellido (3)}
\pertenenciaJurTres{pertenencia (3)}
 
\fechaINICIO{01 de marzo de 2022}		%Fecha de inicio de la cursada de GdP \fechaInicioName
\fechaFINALPlan{19 de abril de 2022} 	%Fecha de final de cursada de GdP
\fechaFINALTrabajo{15 de mayo de 2022}	%Fecha de defensa pública del trabajo final


\begin{document}

\maketitle
\thispagestyle{empty}
\pagebreak


\thispagestyle{empty}
{\setlength{\parskip}{0pt}
\tableofcontents{}
}
\pagebreak


\section*{Registros de cambios}
\label{sec:registro}


\begin{table}[ht]
\label{tab:registro}
\centering
\begin{tabularx}{\linewidth}{@{}|c|X|c|@{}}
\hline
\rowcolor[HTML]{C0C0C0} 
Revisión & \multicolumn{1}{c|}{\cellcolor[HTML]{C0C0C0}Detalles de los cambios realizados} & Fecha      \\ \hline
0      & Creación del documento                                 &\fechaInicioName \\ \hline
1      & Se completa hasta el punto 5 inclusive                 & 15/03/2022 \\ \hline
%2      & Se completa hasta el punto 7 inclusive
%		  Se puede agregar algo más \newline
%		  En distintas líneas \newline
%		  Así                                                    & dd/mm/aaaa \\ \hline
%3      & Se completa hasta el punto 11 inclusive                & dd/mm/aaaa \\ \hline
%4      & Se completa el plan	                                 & dd/mm/aaaa \\ \hline
\end{tabularx}
\end{table}

\pagebreak



\section*{Acta de constitución del proyecto}
\label{sec:acta}

\begin{flushright}
Buenos Aires, \fechaInicioName
\end{flushright}

\vspace{2cm}

Por medio de la presente se acuerda con el Ing. \authorname\hspace{1px} que su Trabajo Final de la \degreename\hspace{1px} se titulará ``\ttitle'', consistirá esencialmente en la implementación de un prototipo de data logger para mediciones meteorológicas, y tendrá un presupuesto preliminar estimado de 600 hs de trabajo y \$ 1000 US, con fecha de inicio \fechaInicioName\hspace{1px} y fecha de presentación pública \fechaFinalName.

Se adjunta a esta acta la planificación inicial.

\vfill

% Esta parte se construye sola con la información que hayan cargado en el preámbulo del documento y no debe modificarla
\begin{table}[ht]
\centering
\begin{tabular}{ccc}
\begin{tabular}[c]{@{}c@{}}Ariel Lutenberg \\ Director posgrado FIUBA\end{tabular} & \hspace{2cm} & \begin{tabular}[c]{@{}c@{}}\clientename \\ \empclientename \end{tabular} \vspace{2.5cm} \\ 
\multicolumn{3}{c}{\begin{tabular}[c]{@{}c@{}} \supname \\ Director del Trabajo Final\end{tabular}} \vspace{2.5cm} \\
%\begin{tabular}[c]{@{}c@{}}\jurunoname \\ Jurado del Trabajo Final\end{tabular}     &  & \begin{tabular}[c]{@{}c@{}}\jurdosname\\ Jurado del Trabajo Final\end{tabular}  \vspace{2.5cm}  \\
%\multicolumn{3}{c}{\begin{tabular}[c]{@{}c@{}} \jurtresname\\ Jurado del Trabajo Final\end{tabular}} \vspace{.5cm}                                                                     
\end{tabular}
\end{table}




\section{1. Descripción técnica-conceptual del proyecto a realizar}
\label{sec:descripcion}

El cambio climático es una de las mayores problemáticas que enfrenta la humanidad en el siglo 21, se estima que 800 millones de personas son vulnerables a los efectos negativos que puede generar este, es por eso que se deben generar políticas enfocadas a luchar y contrarrestar los perjuicios que pueden ser causados por el cambio climático. Estas políticas deben estar basadas en datos y mediciones científicas, las cuales se obtienen en gran parte por medio de estaciones meteorológicas automáticas.   

Una estación meteorológica automática se encarga de tomar mediciones meteorológicas, procesarlas, guardarlas y transmitirlas. El elemento principal de esta es un data logger, que es un computador con distintos periféricos para cumplir las funciones requeridas por la estación. 

En Paraguay, el mercado de estaciones meteorológicas automáticas esta dominado por productos del extranjero, las que tienen costo muy elevado para el presupuesto que manejan las entidades publicas y privadas, también, el bajo nivel de customización que ofrecen, es un impedimento para poder modificar a la estación en base a las posibilidades del cliente.

Actualmente, Tech Enterpise S.A cuenta con data loggers propios, basados en microcontroladores de 8 bits, como el atmega328p y atmega2560. La baja cantidad de periféricos, velocidad, memoria RAM y FLASH con los que cuentan estos microcontroladores, limita las funcionalidades y escalabilidad del código del data logger. Es por esto que la empresa busca migrar a microcontroladores de 32bits, como el STM32F767ZI, que cuenta con una frecuencia máxima de 216MHz, memoria FLASH de 2 Mbytes y memoria SRAM de 512 Kbytes. También, cuenta con interfaz MAC, que permite utilizar integrados PHY para Ethernet.   

En lo que respecta al presente trabajo, se propone diseñar una solución basada en el mencionado microcontrolador, que sea capaz de comunicarse con sensores utilizando el protocolo de comunicación SDI-12, con una memoria FLASH y memoria SD y que pueda transmitir la información por medio de protocolos de capa 2 como el PPP y ethernet.

En la figura 1 se muestra el diagrama en bloques del sistema. El microcontrolador se encarga de comunicarse con distintos periféricos, como la memoria SD, memoria Flash, ADC, etc. Además de poder comunicarse con dichos componentes, debe de ser capaz de actuar como cliente FTP y HTTP.

%\vspace{25px}

\begin{figure}[htpb]
\centering 
\includegraphics[width=.9\textwidth]{./Figuras/diagramaDeConexion.jpeg}
\caption{Diagrama en bloques del sistema.}
\label{fig:diagBloques}
\end{figure}

\vspace{25px}



\section{2. Identificación y análisis de los interesados}
\label{sec:interesados}

\begin{table}[ht]
%\caption{Identificación de los interesados}
%\label{tab:interesados}
\begin{tabularx}{\linewidth}{@{}|l|X|X|l|@{}}
\hline
\rowcolor[HTML]{C0C0C0} 
Rol           & Nombre y Apellido & Organización 	& Puesto 	\\ \hline
Auspiciante   &     Jose Flecha              &     Data system S.A         	&       Director ejecutivo  	\\ \hline
Cliente       & \clientename      &\empclientename	& Director ejecutivo       	\\ \hline
Impulsor      & Julio Mendez    &  CONACYT    	&  Director PROINOVA    	\\ \hline
Responsable   & \authorname       & FIUBA        	& Alumno 	\\ \hline
Orientador    & \supname	      & \pertesupname 	& Director Trabajo final \\ \hline
Opositores    &        Jack  Johnson          &      Campbell Scientific        	&   Sales manager     	\\ \hline
Usuario final & Fernando Pio Barrios   & Dirección de Meteorología e Hidrología              	& Director       	\\ \hline
\end{tabularx}
\end{table}


\section{3. Propósito del proyecto}
\label{sec:proposito}

El propósito de este proyecto es diseñar e implementar el prototipo de un data logger basado en un microcontrolador de 32 bits que permita comunicarse con sensores meteorológicos, procesar los datos, almacenar la información a una memoria SD y trasmitir a un servidor FTP.

\section{4. Alcance del proyecto}
\label{sec:alcance}

El proyecto incluye los siguientes puntos:
\begin{itemize}
\item Diseño, programación e implementación de un sistema embebido que sea capaz de comunicarse con distintos sensores, procesar los datos, guardar y transmitirlos por internet.
\item Diseño de una biblioteca modular en C++, que permita comunicarse con sensores que utilizan el protocolo SDI-12.
\item Diseño de una biblioteca modular en C++, que permita comunicarse con memorias de tipo FLASH.
\item Diseño de la arquitectura del sistema embebido.
\item Prototipo en protoboard del sistema embebido.
\item Diseño de la placa de circuito impreso.
\item Verificación funcional de la placa de circuito impreso.
\end{itemize}

Queda excluido del alcance del proyecto:
\begin{itemize}
\item Desarrollo de un cargador solar.
\item Diseño de un RTOS.
\item Diseño de un stack TCP/IP.
\item Testeo de mas de una semana del sistema.
\item Testeo de rangos de temperatura y humedad mínimos y máximos.
\end{itemize}


\section{5. Supuestos del proyecto}
\label{sec:supuestos}


Para el desarrollo del presente proyecto se supone que: 

\begin{itemize}
	\item Se contara con una placa NUCLEO-F767ZI.
	\item Se podrá contar con los insumos necesarios en tiempo y forma.
	\item Se contara con las instalaciones disponibles para las pruebas pertinentes.
	\item Se dispondrá del tiempo necesario para el desarrollo del proyecto.
	\item Se contara con sensores meteorológicos para realizar las mediciones.
	\item Se contara con módem cellular para conexiones GPRS/GSM.
	\item Se cuenta con toda la información técnica de sensores meteorológicos, modems celulares, microcontroladores, ethernet, etc.
\end{itemize}

\section{6. Requerimientos}
\label{sec:requerimientos}

\begin{enumerate}
	\item Requerimientos de firmware.
		\begin{enumerate}
			\item El proyecto sera desarrollado y mantenido mediante el control de versiones GIT. 
			\item El proyecto sera desarrollado en lenguaje C, compilado con ARM GCC.
			\item Se debera de desarrollar libreria de control del conversor analogico digital.
			\item Se debera de desarrollar libreria de control de la memoria FLASH. 
			\item Se debera de desarrollar libreria del protocolo SDI-12.
		\end{enumerate}
	\item Requerimientos de documentación.
		\begin{enumerate}
			\item El ccodigo sera documentado con la herramienta Doxygen.
			\item El ciclo de vida del proyecto sera respaldado por una memoria tecnica.
		\end{enumerate}
	\item Requerimientos asociados con la arquitectura del sistema.
		\begin{enumerate}
		\item El data logger se debe de basar en el STM32F767ZI.
		\item El STM32F767ZI se debe de comunicar con una memoria FLASH por medio del protocolo SPI.
		\item El STM32F767ZI se debe de comunicar con una memoria SD, por medio del protocolo SPI.
		\item El STM32F767ZI se debe de comunicar con un conversor analogico digital, por medio del protocolo I2C.
		\item El sistema debe de contar con los converores de voltaje para el protocolo SDI-12.
		\end{enumerate}
	\item Requerimientos de la interfaz.
		\begin{enumerate}
		\item El sistema debe de contar con proteccion contra cortocircuitos, sobre voltaje y bajo voltaje. 
		\item El sistema debe de funcionar con un rango de tension entre 7V y 24 VDC.
		\item El sistema debe de contar con dos puertos RS-232.
		\item El sistema debe de contar con un puerto Ethernet.
		\item El sistema debe de contar con una interfaz SD.
		\item El sistema debe de contar con un puerto micro USB.
		\item El sistema debe de contar con borneras para conexion con los sensores.
		\end{enumerate}
\end{enumerate}


\section{7. Historias de usuarios (\textit{Product backlog})}
\label{sec:backlog}

Roles: 
\begin{itemize}
	\item Usuario/a - Persona o entidad donde la informacion proveniente de la estacion sera destinada, 
	\item Instalador/a - Es la persona encargada de la puesta en marcha de la estacion, 
	\item Cliente - Es la persona que auspiciara el proyecto de la estacion
\end{itemize}

Ponderacion:
	\begin{itemize}
		\item Este puntaje esta dado por la relevancia que tiene la historia de usuario para el exito del proyecto definido en esta planificacion.
		\item Se utilizara una escala de prioridad de 1, 2, y 4 puntos para las historias
	\end{itemize}

\begin{consigna}{red}
Descripción: En esta sección se deben incluir las historias de usuarios y su ponderación (\textit{history points}). Recordar que las historias de usuarios son descripciones cortas y simples de una característica contada desde la perspectiva de la persona que desea la nueva capacidad, generalmente un usuario o cliente del sistema. La ponderación es un número entero que representa el tamaño de la historia comparada con otras historias de similar tipo.

El formato propuesto es: "como [rol] quiero [tal cosa] para [tal otra cosa]."

Se debe indicar explícitamente el criterio para calcular los \textit{story points} de cada historia
\end{consigna}

\section{8. Entregables principales del proyecto}
\label{sec:entregables}

\begin{itemize}
	\item Informe final.
	\item Manual de configuracion y puesta en marcha.
	\item Diagrama de circuitos esquemáticos.
	\item Código fuente del firmware.
\end{itemize}


\section{9. Desglose del trabajo en tareas}
\label{sec:wbs}

\begin{enumerate}
\item Planificacion del proyecto (40hs)
	\begin{enumerate}
	\item Realizar el plan de proyecto (20 hs).
	\item Redectar requisitos del firmware (20 hs).
	\end{enumerate}
\item Investigacion preliminar (30 hs)
	\begin{enumerate}
	\item Investigar caracteristicas tecnicas de cada uno de los componentes a utilizar (10 hs)
	\item Investigar librerias TCP/IP (10 hs).
	\item Investigar sobre sistemas operativos de tiempo real (10 hs).
	\end{enumerate}
\item Desarrollo de firmware (200 hs)
	\begin{enumerate}
	\item Configuracion inicial del microcontrolador y sus distintos perifericos. 
	\item Disenar algoritmo de configuracion de la estacion (30 hs).
	\item Disenar libreria para control de memoria FLASH (40 hs).
	\item Integrar liberia para control de memoria SD (10 hs).
	\item Disenar libreria para comunicacion por medio del protocolo SDI-12 (40 hs).
	\item Disenar libreria para control del conversor analogico digital (40 hs).
	\item Disenar liberia para cliente FTP (40 hs).
	\item Integracion de las distintas librerias (40 hs).
	\end{enumerate}
\item Validacion y documentacion (X hs).
	\begin{enumerate}
	\item Ejecutar esquema de pruebas (X hs).
	
	\end{enumerate}
\item Elaboracion de presentacion final (X hs).
	\begin{enumerate}
	\item Redaccion del manuscripto fianl ( hs).
	\end{enumerate}
\end{enumerate}

Cantidad total de horas: (tantas hs)

\section{10. Diagrama de Activity On Node}
\label{sec:AoN}

\begin{consigna}{red}
Armar el AoN a partir del WBS definido en la etapa anterior. 

%La figura \ref{fig:AoN} fue elaborada con el paquete latex tikz y pueden consultar la siguiente referencia \textit{online}:

%\url{https://www.overleaf.com/learn/latex/LaTeX_Graphics_using_TikZ:_A_Tutorial_for_Beginners_(Part_3)\%E2\%80\%94Creating_Flowcharts}

\end{consigna}

\begin{figure}[htpb]
\centering 
\includegraphics[width=.8\textwidth]{./Figuras/AoN.png}
\caption{Diagrama en \textit{Activity on Node}}
\label{fig:AoN}
\end{figure}

Indicar claramente en qué unidades están expresados los tiempos.
De ser necesario indicar los caminos semicríticos y analizar sus tiempos mediante un cuadro.
Es recomendable usar colores y un cuadro indicativo describiendo qué representa cada color, como se muestra en el siguiente ejemplo:



\section{11. Diagrama de Gantt}
\label{sec:gantt}

\begin{consigna}{red}

Existen muchos programas y recursos \textit{online} para hacer diagramas de gantt, entre los cuales destacamos:

\begin{itemize}
\item Planner
\item GanttProject
\item Trello + \textit{plugins}. En el siguiente link hay un tutorial oficial: \\ \url{https://blog.trello.com/es/diagrama-de-gantt-de-un-proyecto}
\item Creately, herramienta online colaborativa. \\\url{https://creately.com/diagram/example/ieb3p3ml/LaTeX}
\item Se puede hacer en latex con el paquete \textit{pgfgantt}\\ \url{http://ctan.dcc.uchile.cl/graphics/pgf/contrib/pgfgantt/pgfgantt.pdf}
\end{itemize}

Pegar acá una captura de pantalla del diagrama de Gantt, cuidando que la letra sea suficientemente grande como para ser legible. 
Si el diagrama queda demasiado ancho, se puede pegar primero la ``tabla'' del Gantt y luego pegar la parte del diagrama de barras del diagrama de Gantt.

Configurar el software para que en la parte de la tabla muestre los códigos del EDT (WBS).\\
Configurar el software para que al lado de cada barra muestre el nombre de cada tarea.\\
Revisar que la fecha de finalización coincida con lo indicado en el Acta Constitutiva.

En la figura \ref{fig:gantt}, se muestra un ejemplo de diagrama de gantt realizado con el paquete de \textit{pgfgantt}. En la plantilla pueden ver el código que lo genera y usarlo de base para construir el propio.

\begin{figure}[htbp]
\begin{center}
\begin{ganttchart}{1}{12}
  \gantttitle{2020}{12} \\
  \gantttitlelist{1,...,12}{1} \\
  \ganttgroup{Group 1}{1}{7} \\
  \ganttbar{Task 1}{1}{2} \\
  \ganttlinkedbar{Task 2}{3}{7} \ganttnewline
  \ganttmilestone{Milestone o hito}{7} \ganttnewline
  \ganttbar{Final Task}{8}{12}
  \ganttlink{elem2}{elem3}
  \ganttlink{elem3}{elem4}
\end{ganttchart}
\end{center}
\caption{Diagrama de gantt de ejemplo}
\label{fig:gantt}
\end{figure}


\begin{landscape}
\begin{figure}[htpb]
\centering 
\includegraphics[height=.85\textheight]{./Figuras/Gantt-2.png}
\caption{Ejemplo de diagrama de Gantt rotado}
\label{fig:diagGantt}
\end{figure}

\end{landscape}

\end{consigna}


\section{12. Presupuesto detallado del proyecto}
\label{sec:presupuesto}

\begin{consigna}{red}
Si el proyecto es complejo entonces separarlo en partes:
\begin{itemize}
	\item Un total global, indicando el subtotal acumulado por cada una de las áreas.
	\item El desglose detallado del subtotal de cada una de las áreas.
\end{itemize}

IMPORTANTE: No olvidarse de considerar los COSTOS INDIRECTOS.

\end{consigna}

\begin{table}[htpb]
\centering
\begin{tabularx}{\linewidth}{@{}|X|c|r|r|@{}}
\hline
\rowcolor[HTML]{C0C0C0} 
\multicolumn{4}{|c|}{\cellcolor[HTML]{C0C0C0}COSTOS DIRECTOS} \\ \hline
\rowcolor[HTML]{C0C0C0} 
Descripción &
  \multicolumn{1}{c|}{\cellcolor[HTML]{C0C0C0}Cantidad} &
  \multicolumn{1}{c|}{\cellcolor[HTML]{C0C0C0}Valor unitario} &
  \multicolumn{1}{c|}{\cellcolor[HTML]{C0C0C0}Valor total} \\ \hline
 &
  \multicolumn{1}{c|}{} &
  \multicolumn{1}{c|}{} &
  \multicolumn{1}{c|}{} \\ \hline
 &
  \multicolumn{1}{c|}{} &
  \multicolumn{1}{c|}{} &
  \multicolumn{1}{c|}{} \\ \hline
\multicolumn{1}{|l|}{} &
   &
   &
   \\ \hline
\multicolumn{1}{|l|}{} &
   &
   &
   \\ \hline
\multicolumn{3}{|c|}{SUBTOTAL} &
  \multicolumn{1}{c|}{} \\ \hline
\rowcolor[HTML]{C0C0C0} 
\multicolumn{4}{|c|}{\cellcolor[HTML]{C0C0C0}COSTOS INDIRECTOS} \\ \hline
\rowcolor[HTML]{C0C0C0} 
Descripción &
  \multicolumn{1}{c|}{\cellcolor[HTML]{C0C0C0}Cantidad} &
  \multicolumn{1}{c|}{\cellcolor[HTML]{C0C0C0}Valor unitario} &
  \multicolumn{1}{c|}{\cellcolor[HTML]{C0C0C0}Valor total} \\ \hline
\multicolumn{1}{|l|}{} &
   &
   &
   \\ \hline
\multicolumn{1}{|l|}{} &
   &
   &
   \\ \hline
\multicolumn{1}{|l|}{} &
   &
   &
   \\ \hline
\multicolumn{3}{|c|}{SUBTOTAL} &
  \multicolumn{1}{c|}{} \\ \hline
\rowcolor[HTML]{C0C0C0}
\multicolumn{3}{|c|}{TOTAL} &
   \\ \hline
\end{tabularx}%
\end{table}


\section{13. Gestión de riesgos}
\label{sec:riesgos}

\begin{consigna}{red}
a) Identificación de los riesgos (al menos cinco) y estimación de sus consecuencias:
 
Riesgo 1: detallar el riesgo (riesgo es algo que si ocurre altera los planes previstos de forma negativa)
\begin{itemize}
	\item Severidad (S): mientras más severo, más alto es el número (usar números del 1 al 10).\\
	Justificar el motivo por el cual se asigna determinado número de severidad (S).
	\item Probabilidad de ocurrencia (O): mientras más probable, más alto es el número (usar del 1 al 10).\\
	Justificar el motivo por el cual se asigna determinado número de (O). 
\end{itemize}   

Riesgo 2:
\begin{itemize}
	\item Severidad (S): 
	\item Ocurrencia (O):
\end{itemize}

Riesgo 3:
\begin{itemize}
	\item Severidad (S): 
	\item Ocurrencia (O):
\end{itemize}


b) Tabla de gestión de riesgos:      (El RPN se calcula como RPN=SxO)

\begin{table}[htpb]
\centering
\begin{tabularx}{\linewidth}{@{}|X|c|c|c|c|c|c|@{}}
\hline
\rowcolor[HTML]{C0C0C0} 
Riesgo & S & O & RPN & S* & O* & RPN* \\ \hline
       &   &   &     &    &    &      \\ \hline
       &   &   &     &    &    &      \\ \hline
       &   &   &     &    &    &      \\ \hline
       &   &   &     &    &    &      \\ \hline
       &   &   &     &    &    &      \\ \hline
\end{tabularx}%
\end{table}

Criterio adoptado: 
Se tomarán medidas de mitigación en los riesgos cuyos números de RPN sean mayores a...

Nota: los valores marcados con (*) en la tabla corresponden luego de haber aplicado la mitigación.

c) Plan de mitigación de los riesgos que originalmente excedían el RPN máximo establecido:
 
Riesgo 1: plan de mitigación (si por el RPN fuera necesario elaborar un plan de mitigación).
  Nueva asignación de S y O, con su respectiva justificación:
  - Severidad (S): mientras más severo, más alto es el número (usar números del 1 al 10).
          Justificar el motivo por el cual se asigna determinado número de severidad (S).
  - Probabilidad de ocurrencia (O): mientras más probable, más alto es el número (usar del 1 al 10).
          Justificar el motivo por el cual se asigna determinado número de (O).

Riesgo 2: plan de mitigación (si por el RPN fuera necesario elaborar un plan de mitigación).
 
Riesgo 3: plan de mitigación (si por el RPN fuera necesario elaborar un plan de mitigación).

\end{consigna}


\section{14. Gestión de la calidad}
\label{sec:calidad}

\begin{consigna}{red}
Para cada uno de los requerimientos del proyecto indique:
\begin{itemize} 
\item Req \#1: copiar acá el requerimiento.

\begin{itemize}
	\item Verificación para confirmar si se cumplió con lo requerido antes de mostrar el sistema al cliente. Detallar 
	\item Validación con el cliente para confirmar que está de acuerdo en que se cumplió con lo requerido. Detallar  
\end{itemize}

\end{itemize}

Tener en cuenta que en este contexto se pueden mencionar simulaciones, cálculos, revisión de hojas de datos, consulta con expertos, mediciones, etc.  Las acciones de verificación suelen considerar al entregable como ``caja blanca'', es decir se conoce en profundidad su funcionamiento interno.  En cambio, las acciones de validación suelen considerar al entregable como ``caja negra'', es decir, que no se conocen los detalles de su funcionamiento interno.

\end{consigna}

\section{15. Procesos de cierre}    
\label{sec:cierre}

\begin{consigna}{red}
Establecer las pautas de trabajo para realizar una reunión final de evaluación del proyecto, tal que contemple las siguientes actividades:

\begin{itemize}
	\item Pautas de trabajo que se seguirán para analizar si se respetó el Plan de Proyecto original:
	 - Indicar quién se ocupará de hacer esto y cuál será el procedimiento a aplicar. 
	\item Identificación de las técnicas y procedimientos útiles e inútiles que se emplearon, y los problemas que surgieron y cómo se solucionaron:
	 - Indicar quién se ocupará de hacer esto y cuál será el procedimiento para dejar registro.
	\item Indicar quién organizará el acto de agradecimiento a todos los interesados, y en especial al equipo de trabajo y colaboradores:
	  - Indicar esto y quién financiará los gastos correspondientes.
\end{itemize}

\end{consigna}


\end{document}
